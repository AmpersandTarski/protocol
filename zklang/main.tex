\documentclass[a4paper]{article}
\usepackage[utf8]{inputenc}
\usepackage{geometry}
\usepackage{amssymb,amsthm,amsmath,amsfonts,longtable,comment,array,ifpdf,hyperref,url}
\usepackage{graphicx}
\usepackage{authblk}

\title{Zero Knowledge Language}
\author[1]{Michael Lodder}
\author[1]{Brent Zundel}
\author[1]{Jason Law}
\author[1]{Lovesh Harchandani}
\author[2]{Jan Camenisch}
\author[2]{Manu Drijvers}

\affil[1]{Evernym}
\affil[2]{IBM Zurich}

\date{10th January 2018, version 0.1}

\begin{document}

\maketitle

\section{Introduction}
Currently there exists many different methods for describing zero knowledge proofs in cryptographic mathematical notation. Zero Knowledge Language or ZKLang attempts to create a formal language to describe basic programmer mappings of zero knowledge proofs to complex cryptographic algorithms.~\footnote{\url{https://docs.google.com/presentation/d/1Zt1oRXASoKo03mnDpG7kNhjMDv6v4GhfIPNO6_aTEYM/edit?usp=sharing}}

\section{Background}

\subsection{Mathmatical symbols}
TODO: Evernym

\subsection{Set Theory}
TODO: Evernym

\subsection{Sigma protocols}
TODO: Evernym

\section{Terminology}
TODO: Both Evernym and IBM
\input{terminology}

\subsection{Mapping Verifiable Claims to ZKLang}
TODO: IBM

\subsection{Mapping the different types to integers}
TODO: IBM

\subsection{Age proof}
TODO: IBM

\subsection{Membership proof}
TODO: IBM

\subsection{CL signatures}
TODO: IBM

\subsection{Pseudonyms,Commitments}
TODO: IBM

\subsection{Range proofs}
TODO: IBM

\subsection{Verifiable Encryption}
TODO: IBM

\subsection{Orchestration}
TODO: IBM

\section{Statements}
TODO: Both Evernym and IBM

\input{statements}

\subsection{Realization of ZKLang Components}
TODO: IBM

\end{document}