\documentclass[a4paper]{article}

\usepackage[utf8]{inputenc}
\usepackage{geometry}
\usepackage{amssymb,amsthm,amsmath,amsfonts,longtable,comment,array,ifpdf,hyperref,url}
\usepackage{graphicx}

\title{Zero Knowledge Language}
\author{Michael Lodder}
\date{4th January 2018, version 0.1}

\begin{document}

\maketitle

\section{Introduction}

Currently there exists many different methods for describing zero knowledge proofs in cryptographic mathematical notation. Zero Knowledge Language or ZKLang attempts to create a formal language to describe basic programmer mappings of zero knowledge proofs to complex cryptographic algorithms.~\footnote{\url{https://docs.google.com/presentation/d/1Zt1oRXASoKo03mnDpG7kNhjMDv6v4GhfIPNO6_aTEYM/edit?usp=sharing}}
\newline
ZKLang requires new terminology which should be easy to understand and expressive to write.
\newline
\newline
\indent $\mathbb{Z}_p$: Multiplicative group modulo \emph{p}. \emph{p} is a prime. The basic rule for computations module a prime is to do the computations using the numbers as integers, but every time result \emph{r} is derived, compute it modulo \emph{p}. Taking the modulo means divide the result \emph{r} by \emph{p}, throw away the quotient, and keep the remainder as the answer. Mathematicians call the set of numbers modulo a prime \emph{p} a \emph{finite field} and often refer to it as "mod p". For example, \emph{p}=13 means the entire field is limited to modulo 13. If \emph{r}=38, then the answer in the group becomes $38\mod{13}=12$. \newline
\newline
\indent $\mathbb{Z}$\textbf{[g, p]}: Multiplicative group with generator \emph{g} modulo \emph{p}. \emph{g} generates the entire group and is called the \emph{primitive element} of the group. \emph{g} generates the set 1, \emph{g}, \emph{$g^2$},\dots,\emph{$g^{q-1}$} where \emph{$g^q$} = $1\mod{p}$. In other words, \emph{$g^q$} is the point where the numbers start to repeat. \emph{g} and \emph{p} are public.\newline
\newline
\indent \textbf{Comm}: Commitment. Hides a value, but ensures that it cannot be changed later. Given the parameters $\mathbb{Z}$[g, p], the statement \textbf{Comm(x)} means  $c=g^x\mod{p}$. The commitment scheme used for zero-knowledge proofs is the \emph{pedersen commitment scheme}. This is given as \textbf{Comm(x, y)} which means $c=g^xh^y\mod{p}$. In this equation, \emph{g} and \emph{h} and generators in \emph{p}, all three are known to all parties. Neither \emph{x} nor \emph{y} can be calculated from \emph{c}. \emph{x} and \emph{y} should chosen so \emph{c} is a prime number.\newline
\newline

\end{document}
