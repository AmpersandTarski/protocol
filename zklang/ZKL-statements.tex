% !TEX root =  main.tex

\subsection{Notation}


\subsection{ZKLang Statements}

At this level, all message $m_i$ are integers.
Terms that the language supports are the following ones.
\begin{align}
&\NIZK\lbrace(m_{i\in h})[m_{i\not\in h},\mathit{ipk},]: \operatorname{Credential}( \mathit{ipk}, m_1, m_2, \ldots, m_n) \rbrace\\
&\NIZK\lbrace()[\emph{nym}]: \operatorname{Nym}( \emph{nym}) \rbrace\\
&\NIZK\lbrace()[\emph{nym},\mathit{scope}]: \operatorname{SNym}( \emph{nym}, \mathit{scope}) \rbrace\\
&\NIZK\lbrace(m)[\emph{epk},\mathit{ctxt}]: \operatorname{Enc}( \emph{epk}, m, \mathit{ctxt}) \rbrace\\
&\NIZK\lbrace(m)[c]: \operatorname{Larger}(m, c) \rbrace\\
&\NIZK\lbrace(m)[c]: \operatorname{Smaller}(m, c) \rbrace \\
&\NIZK\lbrace(m)[c]: \operatorname{Member}(m, c) \rbrace \\
&\NIZK\lbrace(m_{i\in h})[m_{i\not\in h},c]: \operatorname{Com}(m1, m_2, c) \rbrace
\end{align}

\paragraph{Credential $\operatorname{Credential}$.}

\paragraph{Pseudonyms $\operatorname{Nym}$ and $\operatorname{SNym}$.}

\paragraph{Verifiable Encryption  $\operatorname{Enc}$.}

\paragraph{Range Statements $\operatorname{Larger}$ and $\operatorname{Smaller}$.}

\paragraph{Membership $\operatorname{Member}$.}
%%
This express that some attribute is contained in a set represented by the constant $c$.
The value of the constant depends on how the membership proof is realized. 
For instance, the value of $c$ be the set itself, e.g., $\{c_1, c_2, ..., c_l\}$, an cryptographic accumulator, or 
or a signature verification key w.r.t.\ which set element where signed by some authority.


\paragraph{Commitments $\operatorname{Com}$.}
%%
This language element is need to re-use (hidden) attributes in an other protocol. 
For instance if an issuer want to issue a credential that contains an attribute of a user that the issuer shall not be
privy off. 


\paragraph{Possible Extentions}
There are a number of possible extensions that for now are beyond the scope of the language. 
Such extensions include statements on polynomial relations between attributes such as $m_1 = 3*m_2 +m_3^2$.


\subsection{Composing Statements}

Example composition of a statement.
\begin{multline*}
\NIZK\lbrace(m_1,m_2,m_3,m_4)[m_5]: \\
 \operatorname{Credential}( \emph{ipk}_1, m_1, m_2, m_3) \  \wedge \
\and \operatorname{Credential}( \emph{ipk}_2, m_1, m_4, m_5) \  \wedge \\
\and \operatorname{Nym}( \emph{nym}) \  \wedge \
\and \operatorname{Larger}(m_3, c)   \rbrace
\end{multline*}
Explanations of stuff

\subsection{Discussion}

\paragraph{User Secret Keys}

Pseudonyms are always bound to a user secret key, credentials can be bound to a user secret key.
Commitments might also contain a secret key.

If in a composed statement the ownership of several pseudonyms, commitments and credentials are proved,
they must be related to the same secret key. 
Thus, secret keys cannot be mixed and matched.

If credentials are key-bound, they are required to be bound to the same (secret) key. 
