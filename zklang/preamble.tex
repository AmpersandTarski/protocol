\usepackage[utf8]{inputenc}
\usepackage{amssymb,amsthm,amsmath,amsfonts,longtable,comment,array,ifpdf,hyperref,url}
\usepackage{graphicx}
\usepackage{authblk}


\usepackage{float} % additional placement specifier H
\usepackage{subfigure}
\usepackage{tabularx}
\usepackage{amsmath,amsfonts,amssymb,xspace}
\usepackage{booktabs}
\usepackage{comment}
\usepackage{url}
\usepackage{hyperref}
\setcounter{tocdepth}{3}
\usepackage{xspace}
\usepackage{graphicx}
\graphicspath{{figures/}, {pictures/}, {fig/}, {img/}, {image/}, {images/}, {pict/}, {picts/}}
\usepackage{psfrag}
\usepackage{framed}
\usepackage[cm,plain]{fullpage}



% theorem definitions
\newtheorem{theorem}{Theorem}[section]
\newtheorem{corollary}[theorem]{Corollary}
\newtheorem{lemma}[theorem]{Lemma}
% these ones in textrm
\theoremstyle{definition}
\newtheorem{definition}[theorem]{Definition}
\newtheorem{example}[theorem]{Example}
\newtheorem{remark}[theorem]{Remark}
\newtheorem{assumption}{Assumption}

\newcommand{\ceil}[1]{\ensuremath{\left\lceil #1 \right\rceil}}
\newcommand{\set}[1]{\ensuremath{\left\{ #1 \right\}}}
\newcommand{\pmset}[1]{\ensuremath{\pm\left\{ #1 \right\}}}
\newcommand{\ideal}[1]{\ensuremath{\left\langle #1 \right\rangle}}
\newcommand{\Spec}{\ensuremath{\mathcal{S}}}
\newcommand{\lres}{\ensuremath{ \ell_{\mathit{res}} }}
\newcommand{\lszk}{\ensuremath{ \ell_{\text{\o}} } }

\newcommand{\SPK}{\ensuremath{\mathit{SPK}}}
\newcommand{\PK}{\ensuremath{\mathit{PK}}}
\newcommand{\Cal}[1]{\ensuremath{\mathcal{#1}}}
\newcommand{\itemline}{ \item[]$~$ }	% adds a blank line in an item/description env. used for protocols 
\DeclareSymbolFont{AMSb}{U}{msb}{m}{n}
\DeclareMathSymbol{\Ff}{\mathalpha}{AMSb}{"46}
\DeclareMathSymbol{\Bb}{\mathalpha}{AMSb}{"42}
\DeclareMathSymbol{\N}{\mathalpha}{AMSb}{"4E}
\DeclareMathSymbol{\R}{\mathalpha}{AMSb}{"52}
\DeclareMathSymbol{\X}{\mathalpha}{AMSb}{"58}
\DeclareMathSymbol{\Zz}{\mathalpha}{AMSb}{"5A}
\newcommand{\Z}[1]{\ensuremath{\Zz_{#1}} }
\newcommand{\F}[1]{\ensuremath{\Ff_{#1}} }
\newcommand{\Zs}[1]{\ensuremath{\Zz^{\ast}_{#1}}}
\newcommand{\jac}[2]{\ensuremath{\genfrac{(}{)}{0.1ex}{}{#1}{#2}\!}}

\newcommand{\ord}[1]{\mathop{\textrm{ord}}(#1)}
\newcommand{\PP}{\ensuremath{\mathbf{P}}}  % VJS
\newcommand{\TT}{\ensuremath{\mathbf{T}}}  % VJS
\newcommand{\GG}{\ensuremath{\mathbf{G}}}  % VJS
\newcommand{\GGf}{\ensuremath{\mathfrak{G}}}  % VJS
\newcommand{\abs}{\mathop{\mathrm{abs}}}  % VJS
%\newcommand{\PK}{\ensuremath{\mathsf{PK}}}  % VJS
%\newcommand{\SK}{\ensuremath{\mathsf{SK}}}  % VJS
\newcommand{\reject}{\ensuremath{\mathsf{reject}}}  % VJS

\newcommand{\cindist}{\stackrel{c}{\approx}}
\newcommand{\QR}[1]{\ensuremath{\textit{QR}_{#1}}}

\newcommand{\hashkey}{\ensuremath{\mathsf{hk}}}  % VJS
\newcommand{\hash}[1]{\ensuremath{\mathcal{H}_{\hashkey}(#1)}}  % VJS
\newcommand{\hashscheme}{\ensuremath{\mathcal{H}}}  % VJS
\newcommand{\floor}[1]{\ensuremath{\lfloor{#1}\rfloor}}
\newcommand{\de}[1]{\ensuremath{\Delta{#1}}}
\newcommand{\rem}[1]{\ensuremath{\operatorname{rem} #1}}  
\newcommand{\rangefloor}[1]{\ensuremath{[#1]}}  
\newcommand{\range}[1]{\ensuremath{[#1]}}  
\newcommand{\cert}{\ensuremath{\mathit{cert}}}

\newcommand{\apihead}[1]{\noindent\textbf{#1}\\[-3mm]}


% Attributes (name, value, type)
\newcommand{\attName}[1]{\ensuremath{\pi_n(a_#1)}}
\newcommand{\attValue}[1]{\ensuremath{\pi_v(a_#1)}}
\newcommand{\attType}[1]{\ensuremath{\pi_t(a_#1)}}
\newcommand{\mappedValue}[1]{\ensuremath{\pi_{\mathit{mv}}(a_#1)}}
%\newcommand{\simpleValue}[1]{\ensuremath{v_#1}}

% Attribute sets
\newcommand{\Attributes}{\ensuremath{A}}
\newcommand{\AttributesCommitted}{\ensuremath{\Attributes_{c}}}
\newcommand{\AttributesKnown}{\ensuremath{\Attributes_{k}}}
\newcommand{\AttributesHidden}{\ensuremath{\Attributes_{h}}}
\newcommand{\AttributesRevealed}{\ensuremath{\Attributes_{r}}}
\newcommand{\AttributesNonRevealed}{\ensuremath{\Attributes_{\overline{r}}}}
\newcommand{\Identifiers}{\ensuremath{\Cal{I}}}
\newcommand{\IdentifierRevealed}{\ensuremath{\Identifiers_{r}}}
\newcommand{\IdentifierNonRevealed}{\ensuremath{\Identifiers_{\overline{r}}}}

\newcommand{\attributeTestSize}{}
\newcommand{\cred}{\ensuremath{\mathit{cred}}}
\newcommand{\comm}{\ensuremath{\mathit{comm}}}
\newcommand{\rep}{\ensuremath{\mathit{rep}}}
\newcommand{\nym}{\ensuremath{\mathit{nym}}}
\newcommand{\dnym}{\ensuremath{\mathit{dNym}}}
\newcommand{\verEnc}{\ensuremath{\mathit{verEnc}}}
\newcommand{\msg}{\ensuremath{\mathit{msg}}}

%math assignments
\newcommand{\asn}{\ensuremath{\gets}}
% \renewcommand{\equiv}{\equiv}
\newcommand{\becomes}{\ensuremath{:=}}
\newcommand{\inR}{\ensuremath{\in_R}}

\newcommand{\G}{\ensuremath{\mathbb{G}}}
\newcommand{\g}{\ensuremath{\mathrm{g}}}
\newcommand{\bits}{\ensuremath{\{0,1\}}}
\newcommand{\eq}{\ensuremath{=}}
\newcommand{\commit}{\ensuremath{\mathrm{Comm}}}

%Roles: Prover, Verifier, Issuer
\newcommand{\Prover}{{\sc Prover}}
\newcommand{\Verifier}{{\sc Verifier}}
\newcommand{\Issuer}{{\sc Issuer}}
\newcommand{\Recipient}{{\sc Recipient}}

\newcommand{\hashfunction}{\textsf{H}\xspace}
\newcommand{\serializelist}{\textsf{List}\xspace}
\newcommand{\intToBytes}{\textsf{int2B}\xspace}
\newcommand{\stringToBytes}{\textsf{Str2B}\xspace}
\newcommand{\NIZK}{\operatorname{ZKL}}



\floatstyle{boxed}
\newfloat{program}{thp}{lop}
\floatname{program}{Figure}

\def\shortbib {0}

\newcommand{\notimplemented}{\textcolor{red}{\textbf{Not implemented.}}\xspace}


\newcommand{\LRarrow}[2][\veclen]{              %l"ange als optionales argument
  \unitlength1ex
  \begin{picture}(#1,1.25)
  \put(0,0.25){\vector(1,0){#1}}
  \put(0,0.75){\makebox(#1,2){#2}}
  \end{picture}}
\newcommand{\LLarrow}[2][\veclen]{
  \unitlength1ex
  \begin{picture}(#1,1.25)
  \put(#1,0.25){\vector(-1,0){#1}}
  \put(0,0.75){\makebox(#1,2){#2}}
  \end{picture}}

\newcommand{\zos}{\zo^*}  
%\newcommand{\prob}[1]{\operatorname{Pr}[#1]}
\newcommand{\zo}{\{0,1\}}


\newcommand{\changenote}[1]{
 %   {\fontfamily{cmss}\selectfont #1}
  %   \marginpar{\sf $\leftarrow\!\!$change$\!\!\rightarrow$}
 }


\newcommand{\changenotetwo}[1]{
     {\fontfamily{cmss}\selectfont V2.0 #1}
     \marginpar{\sf $\leftarrow\!\!$change$\!\!\rightarrow$}
     }

% \newcommand{\work}[1]{
%      {\fontfamily{cmss}\selectfont todo: #1}
%      \marginpar{\sf $\leftarrow\!\!$work$\!\!\rightarrow$}
%      }
\newcommand{\work}[1]{
     }

% \ShowComments
% \HideComments


% =================================================================
% SPACING

\newcommand{\includeThis}[1]{}
\sloppy

%   \let\oldthebibliography=\thebibliography
%   \let\endoldthebibliography=\endthebibliography
%   \renewenvironment{thebibliography}[1]{%
%     \begin{oldthebibliography}{#1}%
%       \setlength{\parskip}{0ex}%
%       \setlength{\itemsep}{0ex}%
%   }%
%   {%
%     \end{oldthebibliography}%
%   }

% =================================================================



% Notation
\newcommand{\idemix}{\emph{idemix}\xspace}
\newcommand{\eg}{for example}
\newcommand{\ie}{that is}




\newcommand{\imagespacebefore}{}%\vspace{-10pt}}
\newcommand{\imagespaceafter}{}%\vspace{-15pt}}
\newcommand{\imagespacecaption}{\vspace{-15pt}}


\usepackage[british]{babel}
\usepackage[utf8]{inputenc}
\usepackage{graphicx}
\usepackage{amssymb}
\usepackage{amstext}
\usepackage{amsmath}
\usepackage{ae}
\usepackage{multirow}
\usepackage{aecompl}
\usepackage{color}
\usepackage{fancyhdr}
\usepackage{a4}
\usepackage{xspace}
\usepackage{listings}

\definecolor{navy}{rgb}{0, 0, 0.25}
\definecolor{green}{rgb}{0, 0.25, 0}
\definecolor{emph}{rgb}{1,0.3,0}
\definecolor{darkgreen}{rgb}{0.1,0.5,0.15}
\definecolor{steelblue}{rgb}{0.2745,0.5098,0.7059}

\newcommand{\processlist}[3][\relax]{\def\listfinish{#1}\long\def\listact{#2}\processfirst#3\listfinish}
\newcommand{\processfirst}[1]{\ifx\listfinish#1\empty\else\listact{\texttt{#1}}\expandafter\processnext\fi}
\newcommand{\processnext}[1]{\ifx\listfinish#1\empty\else\texttt{,}\listact{\texttt{#1}}\expandafter\processnext\fi}
\newenvironment{method}[3]
{\begin{itemize}\item\texttt{\textcolor{navy}{#1}}\\\texttt{\phantom{..}#2 \textcolor{green}{(\processlist{\\\texttt{\phantom{....}}}{#3}\texttt{)}}}\par}
{\end{itemize}}
\newenvironment{getter}[2]
{\begin{itemize}\item\texttt{\textcolor{navy}{#1}}\\\texttt{\phantom{..}#2 \textcolor{green}{()}}\par}
{\end{itemize}}
\newcommand{\identifier}[1]{\texttt{urn:\allowbreak{}idmx:\allowbreak{}3.0.3:\allowbreak{}block:\allowbreak{}#1}}
\newcommand{\anja}[1]{}%\textcolor{emph}{anja: #1}}
\newcommand{\michael}[1]{}%\textcolor{darkgreen}{michael: #1}}
\newcommand{\gert}[1]{}%\textcolor{blue}{gert: #1}}
\newcommand{\robert}[1]{}%\textcolor{steelblue}{\\\noindent\rule{1ex}{1ex} \textbf{Robert:} #1\\}}
\newcommand{\patrik}[1]{}%\textcolor{grey}{\\\noindent\rule{1ex}{1ex} \textbf{Patrik:} #1\\}}




% % Used for captions after verbatim environment
% \usepackage{float}
% % allows use of "@" in control sequence names
% \makeatletter
% 
% % this creates a custom and simpler ruled box style
% \newcommand\floatc@simplerule[2]{{\@fs@cfont #1 #2 \\[-7pt] \hrule height.4pt depth0pt \kern-2pt}\par}
% \newcommand\fs@simplerule{\def\@fs@cfont{\bfseries}\let\@fs@capt\floatc@simplerule  
%   \def\@fs@pre{\hrule height.8pt depth0pt \kern4pt}%
%   \def\@fs@post{\kern-2pt\hrule height.4pt depth0pt \kern2pt \relax}%
%   \def\@fs@mid{\kern8pt}%
%   \let\@fs@iftopcapt\iftrue}
% 
% % this code block defines the new and custom floatbox float environment
% \newfloat{xml}{thp}{lop}[section]
% \floatname{xml}{XML}


% \makeatletter
% 
% % \newcounter{xml}
% % \newcommand\xmlname{XML}
% % \renewcommand\thexml{\thesection.\@arabic\c@xml}
% % \def\fps@xml{thp}
% % \def\ftype@xml{1}
% % \def\ext@xml{lop}
% % \def\fnum@xml{\xmlname\nobreakspace\thexml}
% \usepackage{float}
% \floatstyle{boxed}
% \newfloat{xmlFloat}{h}{lop}
% \floatname{xmlFloat}{Example}
% \newenvironment{xml}
% {	\xmlFloat
% %     \caption{#2}%
% %     \label{#1}%
%     \verbatim
% }%\@float{xml}}
% {%
%     \endverbatim
%     \vskip-.7\baselineskip
%     \endxmlFloat
% }
% %\end@float}
% 
% \makeatother

\usepackage{verbatim}
\usepackage{float}
\usepackage{listing}
\usepackage{enumitem}

% allows use of "@" in control sequence names
\makeatletter

% this creates a custom and simpler ruled box style
\newcommand\floatc@simplerule[2]{{\@fs@cfont #1 #2 \\[-7pt] \hrule height.4pt depth0pt \kern-2pt}\par}
\newcommand\fs@simplerule{\def\@fs@cfont{\bfseries}\let\@fs@capt\floatc@simplerule
  \def\@fs@pre{\hrule height.8pt depth0pt \kern4pt}%
  \def\@fs@post{\kern-2pt\hrule height.4pt depth0pt \kern2pt \relax}%
  \def\@fs@mid{\kern8pt}%
  \let\@fs@iftopcapt\iftrue}

\makeatletter

\floatstyle{simplerule}
\newfloat{xmlcode}{htp!}{lop}[section]
\floatname{xmlcode}{XML}

\newbox\examplebox
\newenvironment{xml}[2]{%
    \xmlcode
    \caption{#2}%
    \label{#1}%
    \small
	\verbatim%\begin{lstlisting}[language=XML]
}{%
    \endverbatim%\end{lstlisting}
    \normalsize
    \endxmlcode
}


\newenvironment{parameter}[1][\unskip]{%}
%\vspace{0.6em}%
%\noindent\texttt{ #1}\\[3pt]%
%\noindent%
\begin{itemize}[leftmargin=15pt]
  \item[~] \texttt{ #1} \\[2pt]
}{%
\end{itemize}
%\vspace{3pt}\hrule%
%\vspace{0.6em}%
}

\makeatother

