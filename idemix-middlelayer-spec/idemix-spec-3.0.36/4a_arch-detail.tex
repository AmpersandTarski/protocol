\section{Building a Proof}

The ZkDirector, which is a class of the proof engine, coordinates the generation of a proof.
In order to do so it uses the ZkModules that have been provided by the first layer (in particular, the 
proof orchestration). 
These modules expose an interface that encapsulates the functionality and allows the proof engine to be crypto-agnostic. 
The proof engine will execute the following steps:
\begin{itemize}
\item Initialise a ZkBuilderProver object that will keep the proof state as well as provide access to necessary elements such as the system parameters or the secrets manager.
\item Call the method \texttt{initializeModule();}  on all modules that it received as input


\end{itemize}




\section{Pedersen Representation}

In the \emph{initializeModule()} method, which is called by the ZkDirector, all attributes of this representation are registered. 
More concretely, we assume that we need  to prove knowledge of the representation:
\begin{displaymath}
C = \prod_i R_i m_i \pmod n .
\end{displaymath}
In this case, the building block registers the attributes $m_i$ specifying their bit length.
If an attribute is on an external device, the building block will not be able to compute the corresponding contribution.
Therefore, the attribute is registered as external attribute
