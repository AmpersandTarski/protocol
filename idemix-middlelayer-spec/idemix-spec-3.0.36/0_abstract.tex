%!TEX root =  IdmxSpecification.tex

\begin{abstract} 
%
%
As we are transforming into a digital society, it is vital that we protect our data in all our transactions. 
%
We can only meet this goal by not revealing more about ourselves than necessary
as well as requiring that transactions are securely authenticated. 
%
Anonymous credentials (also called private credentials) promise to address both of these seemingly opposing 
requirements at the same time.
%
Anonymous credential systems, a privacy-enhanced public-key infrastructure, are more complex than ordinary 
signature schemes since they provide more functionality in order to address all of the requirements of a 
public-key infrastructure with privacy-protection.
%
Unfortunately, the description of these features are spread over many
research papers and it is often not clear how they could all be integrated
into a single system. 
%
This document describes the Identity Mixer cryptographic library that integrates cryptographic techniques 
from many sources to build an anonymous credential system with a rich feature set. 
%
The library focuses on extensibility and provides extension points to implement further algorithms that realise
the defined functionality.
%
Especially, and in contrast to previous library versions, in addition to the Camenisch-Lysyanskaya signature 
scheme~\cite{camlys02b} the Identity Mixer library implements 
the Brands signature scheme~\cite{brands99}.
%
The interfaces of the library are inspired by the design principles behind the ABC4Trust privacy attribute-based 
credential engine (ABCE).
%
For that reason it can easily be used to serve as library providing the cryptographic operations to the ABCE.
%
While describing the current interfaces of the library, this document also can provide a basis for any standardization 
effort towards a privacy-enhancing public-key infrastructure. 




% The main changes compared to the previous version are:
% \begin{enumerate}
% \item Replacement of the concept of single Master Secret with that of multiple Secrets that can be stored locally or on a smartcard;
% \item Changes to the interface for doing proofs;
% \item Changes to the interface for issuance, so that it is similar to that for doing proofs: this enables Advanced Issuance, i.e., carrying-over attributes or secrets;
% \item Implementation of revocation using accumulators;
% \item Implementation of the Not-Equal predicate;
% \item Serialization for several artifacts;
% \item Support for dynamically generating system parameters;
% \end{enumerate}

%\textit{Advanced issuance} allows attributes are blindly ``carried over'' from existing credentials, carry over the secret key to which a credential is bound from existing pseudonym or credential. The proof specification 
%
%\textit{Accumulator mechanism for revocation.}
%Credentials may need to be revoked for several reasons: the credential and the related secrets may have been compromised, the user may have lost her right to carry a credential, or some of her attribute values may have changed.
%Current implementation allows revocation based on the \emph{revocation handle}, which is a dedicated unique attribute embedded in a credential. 
%%When issuing a revocable credential, the issuer also attaches a witness generated by the revocation authority for the revocation handle of the issued credential.
%This version of the library implements a revocation scheme based on accumulators ([CL02]). Accumulator is used for revocation by using the membership proof for white-listing. The revocation authority accumulates all valid revocation handles into a single value and publishes that value. Users can then show that their credential is valid, by using their witness to prove (in zero-knowledge) that the credential's revocation handle is contained in
%the published white-list accumulator. To avoid updating the accumulator when adding users to the system, we use a slightly modified version, that requires the accumulator to be updated only when revoking users from the system, but not when adding new users to the system.
\end{abstract}